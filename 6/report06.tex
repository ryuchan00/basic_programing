\documentclass[12pt,a4j]{jarticle}
\usepackage{graphicx}
\begin{document}
\title{基礎プログラミングおよび演習 レポート #06}
\author{1920031, 山川竜太郎 (ペア: 氏名・学籍番号または「個人作業」)}
\date{提出日付}
\maketitle

\section{構想・計画・設計}

(どのような構想で絵を生成したか、具体的にどのように計画し、プログラムはどう設計したか)

絵を作成するならドット絵の要領でxとy軸を指定するだけで良い。しかしそれだとコンピューターを使用して計算して絵を出力するという講義の内容から外れてしまうので、自動で絵を生成してランダム性を内包した絵を作成することを目標にした。

\section{プログラムコード}
\begin{verbatim}
(ここにプログラムのソースコードを入れる)
\end{verbatim}
(必ず動作するものを提出してください。また絵を生成するために呼び出すRuby命令を最後の行に追加する。)
(「ruby ファイル名」で実行できるようにするため。)

\section{プログラムの説明}

(プログラムのどの部分が何をしているかを説明する)

\section{生成された絵}

(どんな絵という説明を書くこと)
(絵の画像ファイル形式はPostScriptであること。プログラムと絵が一致してい
ること。ファイル名は適宜変更してよい。)
\begin{center}
\end{center}


\section{考察}

(考察は必須かつ重要。課題をやって分かったことや感想など。)

\section{アンケート}

\subsection{Q1:画像が自由に生成できるようになりましたか。}

(ここにQ1の回答を記入)

\subsection{Q2:画像をうまく生成する「コツ」は何だと思いましたか。}

(ここにQ2の回答を記入)

\subsection{Q3:リフレクション(今回の課題で分かったこと)・感想・要望をどうぞ。}

(ここにQ3の回答を記入)

\end{document}
