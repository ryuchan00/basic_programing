\documentclass[12pt,a4j]{jarticle}
\usepackage{graphicx}
\begin{document}
\title{基礎プログラミングおよび演習 レポート #06}
\author{1920031, 山川竜太郎 (ペア: 1920003:伊東隼人,1720031:倉橋和孝)}
\date{提出日付}
\maketitle

\section{構想・計画・設計}

(どのような構想で絵を生成したか、具体的にどのように計画し、プログラムはどう設計したか)

絵を作成するならドット絵の要領でxとy軸を指定するだけで良い。しかしそれだとコンピューターを使用して計算して絵を出力するという講義の内容から外れてしまうので、自動で絵を生成してランダム性を内包した絵を作成することを目標にした。

\section{プログラムコード}

\verb|packman.rbという名前で作成した。|

\begin{verbatim}
  Pixel = Struct.new(:r, :g, :b)
  $img = Array.new(200) do
    Array.new(300) do
      Pixel.new(255, 255, 255)
    end
  end

  def pset(x, y, r = 0, g = 0, b = 0, a = 0.0)
    if x < 0 || x >= 300 || y < 0 || y >= 200 then
      return
    end
    $img[y][x].r = ($img[y][x].r * a + r * (1.0 - a)).to_i
    $img[y][x].g = ($img[y][x].g * a + g * (1.0 - a)).to_i
    $img[y][x].b = ($img[y][x].b * a + b * (1.0 - a)).to_i
  end

  def writeimage(name)
    open(name, "wb") do |f|
      f.puts("P6\n300 200\n255")
      $img.each do |a|
        a.each do |p|
          f.write(p.to_a.pack("ccc"))
        end
      end
    end
    return true
  end

  def fillcircle(x, y, rad, r = 0, g = 0, b = 0, a = 0.0)
    j0 = (y - rad).to_i
    j1 = (y + rad).to_i
    i0 = (x - rad).to_i
    i1 = (x + rad).to_i

    j0.step(j1) do |j|
      i0.step(i1) do |i|
        if (i - x) ** 2 + (j - y) ** 2 < rad ** 2
          if block_given? then
            yield(i, j)
          else
            pset(i, j, r, g, b, a)
          end
        end
      end
    end
  end

  # rectangleの略
  # w:widthの略、幅という意味
  # h:heightの略、高さという意味
  def fillrect(x, y, w, h, r = 0, g = 0, b = 0, a = 0.0)
    j0 = (y - 0.5 * h).to_i
    j1 = (y + 0.5 * h).to_i
    i0 = (x - 0.5 * w).to_i
    i1 = (x + 0.5 * w).to_i

    j0.step(j1) do |j|
      i0.step(i1) do |i|
        if block_given? then
          yield(i, j)
        else
          pset(i, j, r, g, b, a)
        end
      end
    end
  end

  # 楕円
  def fillellipse(x, y, rx, ry, r = 0, g = 0, b = 0, a = 0.0)
    j0 = (y - ry).to_i
    j1 = (y + ry).to_i
    i0 = (x - rx).to_i
    i1 = (x + rx).to_i

    j0.step(j1) do |j|
      i0.step(i1) do |i|
        if ((i - x).to_f / rx) ** 2 + ((j - y).to_f / ry) ** 2 < 1.0
          if block_given? then
            yield(i, j)
          else
            pset(i, j, r, g, b, a)
          end
        end
      end
    end
  end

  # --- ここからパックマンの敵、ゴーストを作るためのメソッド集

  # 凸型に塗りつぶすメソッド
  # 正方形を4つ並べる
  # x,yは3辺が正方形に面している正方形の対角線の軸を指定する
  #   □
  #  □■□ この黒い正方形のこと
  def fill_convex(x, y, h, r = 0, g = 0, b = 0, a = 0.0)
    fillrect(x, y, h, h, r, g, b, a)
    # 上
    fillrect(x, y - h, h, h, r, g, b, a)
    # 左
    fillrect(x - h, y, h, h, r, g, b, a)
    # 右
    fillrect(x + h, y, h, h, r, g, b, a)
  end

  def two_convex(x, y, w, r = 0, g = 0, b = 0, a = 0.0)
    # 足の太さは4px
    2.times do |i|
      # wを二等分して、その中の1/3の部分にxを打つ
      fill_convex(x - w / 3 + w / 3 * 2 * i, y, 4, r, g, b, a)
    end
  end

  # nはゴーストの数
  def ghosts(n = 1)
    w = 60 # 胴体部分に当たる長方形の横の長さ
    h = 30 # 胴体部分に当たる長方形の縦の長さ
    s = 4 # ゴーストの両足に当たる凸を構成する正方形4つのうちの1辺の長さ
    rx = 6
    ry = 8

    # 赤、ピンク、水色、オレンジ
    red = [255, 0, 0] # 赤
    pink = [255, 182, 193] # ピンク
    light_blue = [0, 255, 255]
    orange = [255, 165, 0]

    n.times do
      x = rand((w / 2)..(300 - w / 2)) # 胴体部分に当たる長方形のx軸
      y = rand((10 + h / 2)..(200 - h / 2)) # 胴体部分に当たる長方形のy軸
      color = [red, pink, light_blue, orange].sample

      # まずゴーストの胴体である長方形を配置する
      fillrect(x, y, w, h, color[0], color[1], color[2], 0.0)
      # ゴーストの頭となるような楕円を配置する
      fillellipse(x, y - h / 2, w / 2, 10, color[0], color[1], color[2], 0.0)
      # 足を切り抜く
      two_convex(x, y + h / 2 - s / 2, w, 255, 255, 255, 0.0)
      fillrect(x, y + h / 2 - s, s * 2, s * 2, 255, 255, 255, 0.0)
      # ゴーストの目となるような楕円を配置する
      # 左目
      fillellipse(x - w / 4, y - h / 4, rx, ry, 255, 255, 255, 0.0)
      # 目玉
      fillcircle(x - w / 4 - rx / 2, y - h / 4 + ry / 3, 3, 0, 0, 0, 0.0)
      # 右目
      fillellipse(x + w / 4, y - h / 4, rx, ry, 255, 255, 255, 0.0)
      # 目玉
      fillcircle(x + w / 4 - rx / 2, y - h / 4 + ry / 3, 3, 0, 0, 0, 0.0)
    end
    writeimage(__FILE__.match(%{(^.*).rb})[1] + ".ppm")
  end

  ghosts 10

  # 実行方法は ruby packman.rbを実行する

\end{verbatim}
(必ず動作するものを提出してください。また絵を生成するために呼び出すRuby命令を最後の行に追加する。)
(「ruby ファイル名」で実行できるようにするため。)

\section{プログラムの説明}

(プログラムのどの部分が何をしているかを説明する)

\section{生成された絵}

(どんな絵という説明を書くこと)
(絵の画像ファイル形式はPostScriptであること。プログラムと絵が一致してい
ること。ファイル名は適宜変更してよい。)
\begin{center}
\end{center}


\section{考察}

(考察は必須かつ重要。課題をやって分かったことや感想など。)

\section{アンケート}

\subsection{Q1:画像が自由に生成できるようになりましたか。}

(ここにQ1の回答を記入)

\subsection{Q2:画像をうまく生成する「コツ」は何だと思いましたか。}

(ここにQ2の回答を記入)

\subsection{Q3:リフレクション(今回の課題で分かったこと)・感想・要望をどうぞ。}

(ここにQ3の回答を記入)

\end{document}
